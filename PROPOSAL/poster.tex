\PassOptionsToPackage{unicode}{hyperref}
\documentclass[t]{beamer}

\usepackage[
  orientation=portrait,
  size=a0,
  scale=1.0,
]{beamerposter}

\usetheme{tudoposter}

\usepackage{fontspec}

%\usepackage{polyglossia}
%\setmainlanguage{german}
\usepackage[english]{babel}

\usepackage{csquotes}
\usepackage{microtype}
\usepackage{mathtools}

\usepackage{blindtext}

\usepackage{multicol}
\setlength{\columnsep}{1em}

\usepackage{xfrac}

\usepackage{tikz}
\usetikzlibrary{shapes, arrows, backgrounds, fit, tikzmark, arrows.meta, calc, quotes, angles}

\usepackage[center]{caption}

\usepackage{siunitx}

\usepackage[export]{adjustbox}

\usepackage{xcolor}
\usepackage{listings}
\lstset{
  commentstyle=\color{black!60},    % comment style
}

\lstdefinestyle{base}{
  moredelim=**[is][\color{black!30}]{@}{@},
}

\usepackage{mdframed} %nice frames
\definecolor{light-gray}{gray}{0.95} %the shade of grey that stack exchange uses

\usepackage{graphicx}

% Evaluate at command
\NewDocumentCommand{\evalat}{sO{\big}mm}{%
  \IfBooleanTF{#1}
   {\mleft. #3 \mright|_{#4}}
   {#3#2|_{#4}}%
}


%% this is used to create an inline bibliography
\usepackage[backend=biber, style=numeric, sorting=none]{biblatex}
\addbibresource{biblatex-phys.bib}

\DeclareFieldFormat*{title}{\textit{#1}}
\renewcommand*{\bibfont}{\footnotesize}
\defbibenvironment{bibliography}
  {\noindent}
  {\unspace}
  {}

\renewbibmacro*{begentry}{%
  \usebeamercolor{bibliography item}%
  \color{bibliography item.fg}%
  \printtext[labelnumberwidth]{%
    \printfield{prefixnumber}%
    \printfield{labelnumber}%
  }%
  \setunit{\addnbspace}%
}
\renewcommand*{\finentrypunct}{\addperiod\space}

\newlength{\thirdtextwidth}
\setlength\thirdtextwidth{0.333333\textwidth}

\newlength{\itemseparation}
\setlength\itemseparation{0.25em}

\title{High-energy lepton and photon propagation with the simulation framework PROPOSAL}
\author{Jean-Marco Alameddine$^{1}$, Pascal Gutjahr$^{1}$}
\institute[ETH]{$^{1}$TU Dortmund University, Otto-Hahn-Str. 4a, 44227 Dortmund, Germany}
\date{4. Juli 2015}

\titlegraphic{%
  \includegraphics[width=1.0\linewidth, left]{tudo.pdf}\\
  %\vspace{0.2em}
  %\includegraphics[width=0.8\linewidth, left]{BUW.png}\\
  %\vspace{0.2em}
  %\includegraphics[width=1.0\linewidth, left]{dfg_logo_englisch_blau_en.jpg}
}

\begin{document}


  \begin{columns}[onlytextwidth]%
    \begin{column}{0.45\textwidth}%
      \begin{block}[equal height group=F]{Introduction to PROPOSAL}%
  \begin{itemize}
    \setlength\itemsep{\itemseparation}
    \item Monte Carlo simulations are crucial to train machine learning algorithms
    \begin{itemize}
      \setlength\itemsep{\itemseparation}
      \item[$\rightarrow$] The underlying tools need to be both precise and performant
    \end{itemize}
    \item PROPOSAL is a simulation framework, providing 3D Monte Carlo simulations of high-energy electrons, positrons, muons, taus and photons
    \item Different parametrizations of physical processes, including up-to-date parametrizations, are available
    \item High-performance and high-precision simulations, optimized for large-scale particle propagation
  \end{itemize}

      \vspace{1em}
              \begin{center}
                \colorbox{light-gray}{
            \begin{minipage}[ht]{0.75\linewidth}
              \begin{center}
              \textbf{Find the PROPOSAL repository under:}\\ \url{github.com/tudo-astroparticlephysics/PROPOSAL} %\vspace{0.5em}\\
              \end{center}
            \end{minipage}
            \begin{minipage}[ht]{0.24\linewidth}
              \centering
                \includegraphics[width=0.66\linewidth, valign=t]{plots/qr_proposal_transparent.png}
            \end{minipage}
                }
              \end{center}

      \end{block}%
    \end{column}%

    \begin{column}{0.55\textwidth}%
      \begin{block}[equal height group=F]{How to use PROPOSAL}%
        \begin{itemize}
          \setlength\itemsep{\itemseparation}
          \item PROPOSAL can be used as a C\texttt{++} or a Python library
            \begin{itemize}
              \setlength\itemsep{\itemseparation}
              \item[$\rightarrow$] Simple Python installation with \colorbox{tuYellow}{\texttt{pip install proposal}}
              \item[$\rightarrow$] C\texttt{++} installation using the package manager Conan and CMake
            \end{itemize}
          \item Information about the configuration environment can be read using a JSON file
        \end{itemize}
        \vspace{-0.75em}
        \begin{columns}[onlytextwidth]
        \begin{column}{0.48\textwidth}
        \begin{mdframed}[backgroundcolor=light-gray, roundcorner=10pt,leftmargin=1, rightmargin=1, innerleftmargin=15, innertopmargin=15,innerbottommargin=15, outerlinewidth=1, linecolor=light-gray]

          \lstinputlisting[
          language=Python,
          basicstyle=\footnotesize\ttfamily,
          style=base,
          escapechar=\$,
          breaklines=true
          ]{code/example.txt}
          \end{mdframed} 

          \end{column}
          \begin{column}{0.07\textwidth}
          \begin{center}
            \vspace{7.5em}
            \begin{tikzpicture}
            \draw[
              -triangle 90,
               line width=4mm,
                postaction={draw, line width=0.7cm, shorten >=1cm, -}
            ] (0,0) -- (2,0);
          \end{tikzpicture}
          \end{center}          
        \end{column}
          \begin{column}{0.45\textwidth}
            \begin{figure}
              \includegraphics[width=\linewidth, height=.4\textheight, keepaspectratio]{plots/example_output.pdf}
            \end{figure}
          \end{column}
        \end{columns}

      \end{block}%
    \end{column}%    
  \end{columns}%


  \begin{columns}[onlytextwidth]%
          \begin{block}[equal height group=J]{Simulation of Deflection Uncertainties on Direction Reconstructions of Muons Using PROPOSAL}%

            \begin{figure}
              \includegraphics[width=0.3\linewidth, height=.4\textheight, keepaspectratio]{plots/fit_median_defl_cut_10percent_only_poly_new_resolution_rescale_no_icecube_paper_final_all.pdf}
            \end{figure}

          \end{block}
  \end{columns}%
  \begin{columns}[onlytextwidth]%
    \begin{column}{\thirdtextwidth}%
      \begin{block}[equal height group=B]{Application: CORSIKA~8}%
              \begin{itemize}
                \setlength\itemsep{\itemseparation}
                \item New version of the air shower simulation framework CORSIKA
                \begin{itemize}
                  \setlength\itemsep{\itemseparation}
                  \item[$\rightarrow$] Entirely new code structure, based on modern C\texttt{++} 
                  \item[$\rightarrow$] Focus on flexibility, modularity, efficiency and reliability \cite{Engel2018}
                \end{itemize}
                \item PROPOSAL is used to simulate the electromagnetic and muonic shower component
                \begin{itemize}
                  \setlength\itemsep{\itemseparation}
                  \item[$\rightarrow$] PROPOSAL provides individual modules, where each module solves specific physical tasks \cite{Alameddine_2020}
                  \item[$\rightarrow$] CORSIKA~8 uses these modules to calculate interaction lengths, energy losses, multiple scattering and secondary particles 
                \end{itemize}
                \item First comparisons of CORSIKA~8 and CORSIKA~7: Good agreement for simulations of electromagnetic showers \cite{Alameddine:2021iq}
              \end{itemize}

        \vspace{0.5em}

              \begin{figure}
                \centering
                \includegraphics[width=0.8\linewidth, height=.4\textheight, keepaspectratio]{plots/shower_horizonal.png}
                \caption*{\SI{1}{\tera\electronvolt} $e^-$ shower simulated with CORSIKA~8}
              \end{figure}

              %\begin{itemize}
              %  \item[$\rightarrow$] More on CORSIKA~8 in talk by A.~Sandrock (Thursday, 5:30 PM)
              %\end{itemize}

      \end{block}%
    \end{column}%
    \begin{column}{\thirdtextwidth}%
      \begin{block}[equal height group=B]{Application: Neutrino telescopes}%
        \begin{itemize}
          \setlength\itemsep{\itemseparation}
          \item PROPOSAL is used by neutrino telescopes, for example in the IceCube Neutrino observatory or in RNO-G
          \item Simulation of muon and tau energy losses in ice
          \begin{itemize}
            \setlength\itemsep{\itemseparation}
            \item[$\rightarrow$] Precise simulations and an accurate description of cross sections are crucial
          \end{itemize}
        \end{itemize}

        \vspace{1.75em}

        \begin{figure}
          \centering
          \includegraphics[width=0.7\linewidth, height=.4\textheight, keepaspectratio]{plots/icecube_muon.jpg}
          \caption*{Muon track in the IceCube detector \\(\emph{Source: IceCube Collaboration})}
        \end{figure}

      \end{block}
    \end{column}

    \begin{column}{\thirdtextwidth}%
      \begin{block}[equal height group=B]{Application: Muography}%
        \begin{itemize}
          \setlength\itemsep{\itemseparation}
          \item Non-invasive imaging technique using Cosmic Ray muons
          \item Tracing muon number along trajectories: Provides information, for example on density anomalies
          \item PROPOSAL is a well-suited tool to provide the necessary muon simulations
          \begin{itemize}
            \setlength\itemsep{\itemseparation}
            \item[$\rightarrow$] Currently analyzing the possibilities to use muography in mining with PROPOSAL simulations
          \end{itemize} 
        \end{itemize}
      \vspace{1.25em}
        \begin{figure}
            \begin{tikzpicture}[scale=2.5, every node/.style={scale=0.85}]
                \centering

                \coordinate (A) at (0, 0);

                % ground
                \draw[draw=none, fill=gray, fill opacity=1.0] ($ (A) + (-3.5,-2.5) $) rectangle ++(7, 4);

                % sky
                \draw[draw=none, fill={rgb:red,0.33;green,0.5;blue,0.98}, fill opacity=0.5] ($ (A) + (-3.5,1.5) $) rectangle ++(7, 1);
                \node[draw=none] at ($ (A) + (0.0, 2) $) {Sky};

                % mining shaft
                \draw[draw=none, fill={rgb:black,1;white,2}, fill opacity=1.0] ($ (A) + (2, -2.0) $) rectangle ++(0.25, 3.5);
                \draw[draw=none, fill={rgb:black,1;white,2}, fill opacity=1.0] ($ (A) + (-2.0, -2.0) $) rectangle ++(4, 0.5);

                \node[draw=none] at ($ (A) + (0.8, -1.75) $) {Mining shaft};

                % detector
                \node [cylinder, shape border rotate=90, draw,minimum height=0.40cm,minimum width=0.25cm, aspect=0.4] (detector) at ($ (A) + (-0.5, -1.8) $) {};

                % impurity
                \draw[draw=none,fill=black, fill opacity=0.7] ($ (A) + (0.2, -0.4) $) ellipse (0.3cm and 0.1cm);
                \node[draw=none, text width=1cm] at ($ (A) + (0.8, -0.4) $) {Anomaly};

                % muons
                \draw [densely dotted, blue, line width=0.25mm] ($ (A) + (-2.5, 2.0) $) -- ($ (detector) + (0.2, -0.5) $) node [near start, above, xshift=1ex] (TextNode) {$\mu$};

                \draw [densely dotted, blue, line width=0.25mm] ($ (A) + (-1.0, 2.0) $) -- ($ (detector) + (0.1, -0.5) $) node [near start, above, xshift=1ex] (TextNode) {$\mu$};

                \draw [densely dotted, blue, line width=0.25mm] ($ (A) + (1.0, 2.0) $) -- ($ (A) + (0.2, -0.4) $) node [pos=0.4, above, xshift=-1ex] (TextNode) {$\mu$};

            \end{tikzpicture}
            \caption*{Visualization of the muography technique}
        \end{figure} 

      \end{block}%
    \end{column}%
  \end{columns}%

  \begin{columns}[onlytextwidth]%
    \begin{column}{0.4\textwidth}%
      \begin{block}[equal height group=Z]{Outlook}%
        \begin{itemize}
          \setlength\itemsep{\itemseparation}
          \item Implementation of the LPM effect for inhomogeneous media
          \begin{itemize}
            \setlength\itemsep{\itemseparation}
            \item[$\rightarrow$] Important for very-high-energy air showers
          \end{itemize}
          \item Implementation of only-stochastic propagation
          \begin{itemize}
            \setlength\itemsep{\itemseparation}
            \item[$\rightarrow$] Allows for neutrino propagation with PROPOSAL
          \end{itemize}
        \end{itemize}
      \end{block}
    \end{column}
    \begin{column}{0.4\textwidth}%
      \begin{block}[equal height group=Z]{Contact}%
        \begin{center}
          \begin{figure}[ht]
            \begin{minipage}[ht]{0.75\linewidth}
              \textbf{Find the PROPOSAL repository under:}\\ \url{github.com/tudo-astroparticlephysics/PROPOSAL} \vspace{0.5em}\\
              \textbf{Contact via mail:}\\ \href{mailto:me@jean-marco.alameddine@tu-dortmund.de}{jean-marco.alameddine@tu-dortmund.de} 
            \end{minipage}
            \begin{minipage}[ht]{0.24\linewidth}
              \centering
                \includegraphics[width=0.66\linewidth, valign=t]{plots/qr_proposal.png}
            \end{minipage}
          \end{figure}
        \end{center}
      \end{block}
    \end{column}
    \end{columns}

  \vspace*{\fill}
  \begin{columns}[onlytextwidth]%
    \begin{column}{0.75\textwidth}%
      \begin{alertblock}[equal height group=bottom, fonttitle=\normalsize]{References}
        \begin{multicols}{3}
          \footnotesize%
          \printbibliography%
        \end{multicols}
      \end{alertblock}
    \end{column}
    \begin{column}{0.25\textwidth}%
      \begin{alertblock}[equal height group=bottom, fonttitle=\normalsize]{Acknowledgements}
          \footnotesize%
          This work has been supported by the DFG, Collaborative Research Center SFB 876 (project C3) and Collaborative Research Center SFB 1491 as well as by the BMBF, project 05A20PEA.\\
          Furthermore, we acknowledge funding by the DFG under the grant number SA 3876/2-1.
      \end{alertblock}
    \end{column}
  \end{columns}

  %\begin{block}[equal height group=bottom, fonttitle=\normalsize]{References}
  %  \begin{multicols}{3}
  %    \footnotesize%
  %    \printbibliography%
  %  \end{multicols}
  %\end{block}
\end{document}
